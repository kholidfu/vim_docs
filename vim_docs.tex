\documentclass{article}

\usepackage{graphicx}
\usepackage{hyperref}
\hypersetup{
    colorlinks,
    citecolor=blue,
    filecolor=blue,
    linkcolor=blue,
    urlcolor=red
}
\usepackage{url}
% verbatim small
\makeatletter
\g@addto@macro\@verbatim\footnotesize
%\renewcommand{\l@section}{\@dottedtocline{1}{1.5em}{2.6em}}
\renewcommand{\l@subsection}{\@dottedtocline{2}{1.5em}{3.0em}}
%\renewcommand{\l@subsubsection}{\@dottedtocline{3}{7.4em}{4.5em}}
\makeatother

\author{
  @sopier\footnote{\url{http://twitter.com/sopier}}\\
  \footnotesize{\url{https://github.com/sopier/vim\_docs}}
}
\title{Tutorial Vim}
\date{}

\begin{document}
\maketitle
\tableofcontents
\pagebreak

\section{Pendahuluan}
\subsection{Tentang Tutorial Ini}
Tulisan-tulisan ini sebenarnya bukanlah tutorial lengkap
yang mengajarkan kepada Anda seluk-beluk program Vim,
melainkan sekedar catatan pribadi penulis yang coba
dituangkan kedalam sebuah berkas elektronik dengan tujuan
untuk dokumentasi pribadi, syukur-syukur kalau ada pihak
lain yang membacanya dan mampu mendapatkan manfaat dari
tulisan-tulisan ini.

\subsection{Tentang Vim}
Vim adalah sebuah program penyunting teks yang dapat
dikonfigurasi sesuai kebutuhan sehingga proses penyuntingan
teks menjadi efisien. Vim merupakan versi vi yang
terbarukan dan disebarluaskan pada hampir semua sistem
UNIX.\footnote{http://www.vim.org}

Vim sering dijuluki sebagai program 'penyunting bagi
\emph{programmer}', dan memang kenyataannya sangat berguna
untuk kebutuhan \emph{programming}, sehingga beberapa
menyebutnya sebagai sebuah IDE (\emph{Integrated Development
Environment}). Kenyataannya, Vim tidak hanya digunakan para
\emph{programmer}. Vim sesuai digunakan untuk semua
kebutuhan yang berkaitan dengan penyuntingan teks, dari
menulis \emph{email} sampai penyuntingan berkas konfigurasi.

Vim bukanlah program pemroses kata (\emph{word processor}),
dan tidak memiliki fitur WYSIWYG, Vim adalah sebuah alat
penyunting yang harus dipelajari cara penggunaannya.
Pertanyaan selanjutnya adalah sesulit apa sih belajar Vim
itu? Sejauh pengetahuan penulis mempelajari Vim itu tidak
akan ada habisnya, karena tujuan utama kita adalah efisiensi.
Sesuatu yang hari ini dirasa sudah efisien belum tentu esok
masih tetap efisien. Ini juga berarti proses \emph{continous
improvement} berjalan terus-menerus. Hal ini juga berkaitan
dengan seberapa kreatif si pengguna dalam mengoptimalkan
Vim.

Vim adalah \emph{charityware}, dan memiliki lisensi
\emph{GPL-compatible}, sehingga dapat didistribusikan secara
bebas dan gratis, namun jika Anda merasa mendapatkan manfaat
dari Vim, para pengembang Vim meminta Anda untuk membantu
dengan melakukan donasi dengan membantu anak-anak di Uganda
melalui ICCF.\footnote{http://iccf-holland.org/}

Perlu dicatat bahwa penggunaan Vim akan optimal dalam
kondisi \emph{mouseless} operation, artinya penggunaan
perangkat \emph{mouse} hanya akan memperlambat kerja kita,
untuk itu mari kita biasakan bekerja tanpa \emph{mouse}. Ini
juga menjadi salah satu alasan kenapa banyak yang bilang
belajar Vim itu sulit. Namun jangan khawatir, untuk bisa
produktif dan efisien, Anda tidak perlu menguasai semua
konsep Vim, cukup kuasai beberapa konsep dasar, setelah itu
konsep lain dapat Anda pelajari sesuai kebutuhan.

\section{Penyuntingan Dasar}

Sebelum menjalankan Vim, tentunya Anda harus \emph{install}
dulu programnya, jika Anda menggunakan sistem operasi
Ubuntu, Anda dapat meng-\emph{install} menggunakan
perintah:\footnote{Penulis menggunakan sistem operasi Ubuntu
12.04 dan aplikasi xterm sebagai \emph{terminal emulator}}

\begin{verbatim}
$ sudo apt-get install vim
\end{verbatim}

Setelah Vim ter-\emph{install}, hal penting dan mendasar
sebenarnya adalah membuat berkas \verb=.vimrc= pada
direktori \verb=home= Anda. Berkas ini berguna untuk
konfigurasi program Vim, namun berhubung materi tersebut
masuk pada tingkatan menengah, mari kita jalankan saja dulu
Vim dengan konfigurasi \emph{default}:

\begin{verbatim}
$ vim contoh.txt
\end{verbatim}

Anda sebenarnya juga dapat meng-\emph{install} \verb=gVim=,
versi \emph{graphical} dari Vim dengan fitur \emph{menu}
pada bagian atas \emph{window} sehingga dapat diakses dengan
perangkat \emph{mouse}.  Namun sekali lagi, tutorial ini
mengasumsikan kita tidak menggunakan bantuan perangkat
\emph{mouse} sama sekali, sehingga aplikasi Vim pun sudah
cukup.

Setelah menjalankan perintah di atas, Anda akan mendapatkan
tampilan kurang lebih sebagai berikut:

\vspace{12pt}

\includegraphics[scale=0.6]{vim1.jpg}

\vspace{12pt}

Mulai dari sini Anda mungkin mulai bingung, bagaimana cara
kita mengetik? Perlu diketahui bahwa Vim mengenal 3
\emph{mode} yang harus kita mengerti dulu, yakni
\emph{Insert Mode}, \emph{Normal Mode} dan 
\emph{Insert Mode}. Secara \emph{default}, Vim berada pada
\emph{Command Mode}, dan Anda diharapkan selalu kembali ke
\emph{mode} ini setiap Anda selesai melakukan penyuntingan
teks. 

Mari kita pahami dulu arti dari masing-masing \emph{mode}
tersebut:

\begin{itemize}
    \item \emph{Normal Mode} adalah \emph{mode} di mana Anda
        dapat menjalankan perintah (\emph{command}). 
        \emph{Mode} ini merupakan \emph{mode} ketika Anda
        menjalankan Vim pertama kali.
    \item \emph{Insert Mode} adalah \emph{mode} di mana Anda
        memasukkan teks.
    \item \emph{Visual Mode} adalah \emph{mode} di mana Anda
        dapat menyorot secara \emph{visual} sekumpulan teks,
        sehingga Andapat melakukan operasi penyuntingan pada
        teks tersebut.
\end{itemize}

Untuk memulai proses pengetikan, tekan tombol \verb=i=, dan
kalau Anda perhatikan bagian bawah layar Vim Anda akan
berubah menjadi \verb=-- INSERT --=

\vspace{12pt}

\includegraphics[scale=0.6]{vim2.jpg}

\vspace{12pt}

Sekarang Anda dapat mulai mengetik, misal \verb=Hello World!=

Setiap selesai mengetik, Anda disarankan untuk kembali ke
\emph{command mode}, hal ini dapat dilakukan dengan menekan
tombol \verb=Esc=\footnote{Selain Esc, Anda dapat juga
    menggunakan kombinasi Ctrl+c atau Ctrl+[ atau Anda dapat
    mengkonfigurasi pilihan tombol sesuai selera Anda dengan
membuat konfigurasi berkas .vimrc yang akan dibahas pada
bagian lebih lanjut}. Perhatikan sekarang tanda 
\verb=-- INSERT MODE --= sekali lagi menghilang, dan ini
berarti Anda berada pada \emph{command mode}. 

\vspace{12pt}

\includegraphics[scale=0.6]{vim3.jpg}

\vspace{12pt}

Sekarang simpan berkas ini dengan mengetik \verb=:w=, yang
apabila berhasil, tanda \verb=[+]= di sebelah kanan nama
berkas menghilang.

\vspace{12pt}

\includegraphics[scale=0.6]{vim4.jpg}

\vspace{12pt}

Masih dalam \emph{command mode}, sekarang tekan \verb=:q=
untuk keluar (\emph{exit}) dari Vim.

\vspace{12pt}

\includegraphics[scale=0.6]{vim5.jpg}

\vspace{12pt}

Mari kita rangkum, perintah-perintah yang sudah Anda lakukan
di atas:

\begin{verbatim}
i   =>   beralih ke insert mode
Esc =>   beralih ke command mode
:w  =>   menyimpan berkas
:q  =>   keluar dari Vim
\end{verbatim}

Selamat! Anda baru saja membuat berkas \verb=contoh.txt= dan
mengisikan kalimat \verb=Hello World!= ke dalam berkas
tersebut! Memang, pada awalnya terasa rumit, namun ketika
Anda sudah terbiasa dan mulai menguasai beberapa kombinasi
tombol perintah lebih lanjut, lambat laun Anda akan merasakan apa
yang dimaksud dengan ``\emph{The power of Vim}``.

\section{Navigasi}
Seperti disebutkan di atas bahwa penggunaan Vim yang efisien
adalah dengan meniadakan penggunaan perangkat \emph{mouse}
dan mengoptimalkan penggunaan perangkat \emph{keyboard}. Hal
ini juga berarti bahwa Anda diharapkan mengetahui bagaimana
mengetik dengan \emph{keyboard} secara efisien. 

Salah satu teori yang sering dipakai adalah \emph{Home Row
Technique}, yakni menempatkan keempat jari kiri pada tombol
A, S, D, F dan keempat jari kanan pada tombol J, K, L, ;.
Dengan teknik ini, Anda dapat menjangkau tombol lain, baik
yang berada di atas maupun di bawah tombol-tombol di atas
secara lebih baik.

Hebatnya Vim juga menggunakan teori tersebut secara
intensif, daripada menggunakan tombol panah yang biasanya
terletak pada sebelah kanan \emph{keyboard} dan sulit
dijangkau, Vim menggunakan tombol H, J, K, L untuk navigasi
ke kiri, ke bawah, ke atas, dan ke kanan. 

\begin{verbatim}
h   =>   Bergerak ke kiri satu karakter
j   =>   Bergerak ke bawah satu baris
k   =>   Bergerak ke atas satu baris
l   =>   Bergerak ke kanan satu karakter
\end{verbatim}

Memang pada awalnya perlu waktu untuk penyesuaian, namun
ketika sudah mahir, lihatlah bagaimana kecepatan mengetik
Anda dapat meningkat secara signifikan karena Anda tidak
perlu `menggeser` tangan kanan Anda untuk mengjangkau tombol
panah.

Anda dapat mengkombinasikan tombol navigasi di atas dengan
angka untuk melakukan navigasi secara lebih cepat, misal:

\begin{verbatim}
10h =>   ke kiri 10 karakter
15j =>   ke bawah 15 baris
5k  =>   ke atas 5 baris
4l  =>   ke kanan 4 karakter
\end{verbatim}

dan beberapa kombinasi tombol lainnya:

\begin{verbatim}
Ctrl+F  =>   menuju ke bawah satu layar
Ctrl+B  =>   menuju ke atas satu layar
Ctrl+D  =>   menuju ke bawah setengah layar
Ctrl+U  =>   menuju ke atas setengah layar
M       =>   menuju ke bagian tengah dari layar
H       =>   menuju ke bagian paling atas dari layar
L       =>   menuju ke bagian paling bawah dari layar
\end{verbatim}

Bila sudah mahir, Anda dapat mencoba melakukan navigasi
berdasar kata (\emph{word}), misal:

\begin{verbatim}
w   =>   menuju ke kata berikutnya
W   =>   menuju ke kata berikutnya (berdasar spasi)
b   =>   menuju ke kata sebelumnya
b   =>   menuju ke kata sebelumnya (berdasar spasi)
e   =>   menuju ke akhir kata
E   =>   menuju ke akhir kata (berdasar spasi)
ge  =>   menuju ke akhir kata sebelumnya
gE  =>   menuju ke akhir kata sebelumnya (berdasar spasi)
8w  =>   menuju ke kata kedelapan dari posisi kursor sekarang
5b  =>   menuju ke kiri 5 kata dari posisi kata sekarang
3e  =>   menuju ke akhir karakter dari kata ketiga
\end{verbatim}

Pertanyaannya apa beda \verb=w= dengan \verb=W= (atau
\verb=b= dengan \verb=B=)?

Navigasi dengan \verb=w= masih memperhitungkan tanda
\emph{punctuation} seperti \verb=,.:/!= dll, sedangkan
\verb=W= hanya memperhitungkan spasi. Ambil contoh kalimat
berikut:

\begin{verbatim}
http://www.infotiket.com menyediakan layanan info tiket
pesawat, konser, kereta api dan hotel
\end{verbatim}

Ketik perintah berikut:

\begin{verbatim}
^   =>   berpindah ke awal baris
w   =>   berpindah ke :
W   =>   kursor bergerak ke huruf m pada kata 'menyediakan'
\end{verbatim}

Anda dapat juga melakukan navigasi menggunakan tombol
\verb=f= untuk pencarian ke kanan dan \verb=F= untuk
pencarian ke kiri. Sebagai contoh kalimat berikut:

\begin{verbatim}
Saya sedang belajar Vim
\end{verbatim}

Kursor Anda sekarang berada pada awal kalimat, dan Anda
ingin menuju ke kata Vim, Anda dapat melakukannya dengan
kombinasi tombol \verb=4w= atau Anda dapat juga dengan
menggunakan perintah:

\begin{verbatim}
fV
\end{verbatim}

Tombol di atas berarti cari ke arah kanan karakter \verb=V=.
Sekarang kursor Anda dengan cepat berpindah ke huruf \verb=V=.
Perlu diingat tombol pencarian ini bersifat \emph{case
sensitif}, sehingga pada contoh di atas karena yang kita
cari adalah huruf \verb=V= kapital, maka kita harus menekan tombol
pada \verb=keyboard= \verb=f= diikuti dengan \verb=Shift+v=.

Untuk pencarian ke kiri, gunakan huruf \verb=F= (f kapital),
diikuti dengan huruf yang ingin Anda cari. Anda dapat
mengulangi pencarian berikutnya dengan menekan tombol
\verb=;=. Misal pada kalimat berikut:

\begin{verbatim}
Tutorial Belajar Vim Versi 0.0.8 (Vladimir)
\end{verbatim}

Untuk menuju ke kata \verb=Vladimir=, gunakan kombinasi tombol
berikut:

\begin{verbatim}
fV;;
\end{verbatim}

dan kursor Anda pun dengan cepat berpindah ke huruf \verb=V= pada
kata \verb=Vladimir=. Jika Anda ingin kembali ke huruf \verb=V=
sebelumnya, tekan tombol koma (\verb=,=).

\begin{verbatim}
^fV;;,,
\end{verbatim}

Perintah di atas mencari huruf \verb=V= pada kata \verb=Vladimir=,
kemudian kembali ke huruf \verb=V= pada kata \verb=Vim=.

Perlu diingat bahwa tombol \verb=f= ini hanya dapat
digunakan untuk mencari karakter pada satu baris saja, dan
sifatnya lebih kepada navigasi daripada fungsi pencarian.
Untuk fungsi pencarian kata pada dokumen, akan dibahas pada
bagian lain.

Selain fitur \verb=f=, Vim juga memiliki fitur tombol
\verb=t= yang berfungsi seperti \verb=f=, bedanya \verb=t=
akan bergerak ke 1 karakter sebelum karakter yang dicari.
Untuk lebih jelasnya perhatikan contoh berikut:

Anda memiliki kalimat:

\begin{verbatim}
Vim merupakan program penyunting teks yang hebat.
\end{verbatim}

Anda ingin mencari 1 karakter sebelum huruf \verb=a= pada
kata \verb=hebat= (yakni huruf \verb=b=). Berikut kombinasi
tombol yang dapat Anda lakukan.

\begin{verbatim}
^
ta;;;;
\end{verbatim}

Selanjutnya navigasi untuk menuju ke awal atau akhir baris,
dapat digunakan tombol:

\begin{verbatim}
0   =>   menuju ke awal baris
^   =>   menuju ke karakter pertama dalam sebuah baris
$   =>   menuju ke akhir baris
\end{verbatim}

Menuju ke baris tertentu secara cepat dapat menggunakan
perintah:\footnote{Navigasi menggunakan nomor baris lebih
    saya sukai daripada harus menekan tombol M, H, L
    karena lebih akurat dan praktis, tapi itu semua
tergantung selera Anda..}

\begin{verbatim}
:n      =>   menuju ke baris ke-n
:100    =>   menuju ke baris ke-100
\end{verbatim}

Anda juga dapat melakukan perintah di atas dengan:

\begin{verbatim}
10G     =>   menuju ke baris 10
100G    =>   menuju ke baris 100
\end{verbatim}

Menuju ke awal dokumen atau akhir dokumen:

\begin{verbatim}
gg  =>   menuju ke awal dokumen (baris pertama)
G   =>   menuju ke akhir dokumen (baris terakhir)
\end{verbatim}

Selain kombinasi tombol di atas, kita juga dapat
menggerakkan layar tanpa merubah posisi kursor, sehingga
posisi kursor tepat berada di tengah layar, dengan menekan
tombol:

\begin{verbatim}
zz  =>   reposisi layar sehingga kursor tepat di tengah layar
\end{verbatim}

\section{Fungsi Penyuntingan}

\subsection{Menyisipkan Teks}
Berikut beberapa kombinasi tombol perintah untuk melakukan
penyuntingan teks secara cepat pada Vim:

\begin{verbatim}
i   =>   merubah ke insert mode pada posisi kursor
I   =>   menuju ke kursor kosong pada awal baris dan merubah ke insert mode
a   =>   bergerak ke kanan 1 karakter dan merubah ke insert mode
A   =>   menuju ke kursor kosong pada akhir baris dan merubah ke insert mode
o   =>   membuat baris baru di bawah posisi kursor sekarang dan merubah ke insert mode
O   =>   membuat baris baru di atas posisi kursor sekarang
dan merubah ke insert mode
s   =>   hapus huruf pada posisi kursor dan beralih ke insert mode
S   =>   hapus baris pada posisi kursor dan beralih ke insert mode
\end{verbatim}

Selain kombinasi tombol di atas, Anda juga dapat menggunakan
kombinasi perintah \verb=ct= diikuti dengan karakter 'sampai
dengan' yang Anda tuju, untuk lebih jelasnya perhatikan
contoh berikut:

Semisal Anda memiliki kata \verb=getUrl= yang ingin Anda
menjadi \verb=findUrl=, maka Anda dapat menggunakan perintah
berikut (pastikan posisi kursor berada pada huruf \verb=g=
pada kata \verb=getUrl=):

\begin{verbatim}
ctU     =>   change til U
find    =>   ubah menjadi find
Esc     =>   beralih ke normal mode kembali 
\end{verbatim}

\subsection{Salin dan Tempel}

Kedua fungsi ini saya yakin banyak dipakai ketika kita
sedang bekerja dengan teks, dan Anda mungkin terbiasa
menggunakan perangkat \emph{mouse} untuk melakukan kedua
fungsi tersebut.

Sekali lagi, \emph{mouse is your enemy!}, jadi mari kita
lakukan fungsi tersebut dengan \emph{keyboard way}.  Anda
dapat menyalin sebuah baris dengan mudahnya menggunakan
perintah \verb=yy= atau \verb=Y=, kemudian menuju ke baris
yang Anda inginkan, kemudian tekan \verb=p= untuk
mentempelkan pada baris di bawah kursor, atau tekan \verb=P=
untuk menempelkan pada baris di atas kursor.

Anda dapat juga menempelkan baris tersebut sebanyak yang
Anda inginkan, misal Anda ingin mentempelkan sebanyak 5 kali
pada posisi di bawah kursor Anda sekarang, maka tekan tombol
\verb=5p=, maka otomatis baris yang Anda menyalin akan
tersalin sebanyak 5 kali. 

Contoh:

\begin{verbatim}
Vim is so powerful
\end{verbatim}

Anda ingin menyalin kalimat di atas sebanyak 9 kali, maka
perintah di Vim adalah:

\begin{verbatim}
Y
9p
\end{verbatim}

Hasil:

\begin{verbatim}
Vim is so powerful
Vim is so powerful
Vim is so powerful
Vim is so powerful
Vim is so powerful
Vim is so powerful
Vim is so powerful
Vim is so powerful
Vim is so powerful
Vim is so powerful
\end{verbatim}

Atau Anda ingin menyalin kata \verb=so powerful= saja?

Tempatkan kursor pada huruf \verb=s= pada kata \verb=so=,
kemudian ketik perintah berikut:

\begin{verbatim}
^       =>  ke awal baris
fs;     =>  cari s yang kedua
y2w     =>  salin 2 kata
o       =>  buat baris baru
Esc     =>  beralih ke normal (command) mode
p       =>  put (paste) in here
\end{verbatim}

Hasilnya:

\begin{verbatim}
so powerful
\end{verbatim}

Bagaimana? Sudah mulai merasakan \emph{the power of Vim}??

Mari kita rangkum perintah yang ada dalam bagian ini:

\begin{verbatim}
Y   =>  salin baris
yy  =>  salin baris
yw  =>  salin satu kata di sebelah kanan kursor
yb  =>  salin satu kata di sebelah kiri kursor
y2w =>  salin dua kata di sebelah kanan kursor
p   =>  put hasil salinan
\end{verbatim}

\subsection{Sorot / Visual Mode}

Anda mungkin terbiasa melakukan fungsi sorot ini menggunakan
perangkat \emph{mouse}, dengan melakukan klik kiri, kemudian
\emph{drag} sampai daerah yang Anda inginkan. Vim juga
memungkinkan melakukan itu, meski tentunya proses sorot
menggunakan \emph{keyboard}. 

Vim memiliki 3 bentuk \emph{visual} yang berbeda:

\begin{enumerate}
    \item \emph{per-character visual mode} (v)
    \item \emph{line visual mode} (V)
    \item \emph{block visual mode} (Ctrl+v)
\end{enumerate}

Contoh \emph{per-character visual mode}

Anda memiliki kalimat berikut:

\begin{verbatim}
Vim is great!
\end{verbatim}

Anda dapat menyorot dan mengganti kata \verb=great= menjadi
\verb=superb= dengan melakukan perintah berikut:

\begin{verbatim}
^       =>  menuju ke awal baris
2w      =>  menuju ke kata ke 3
v       =>  aktifkan fungsi sorot
e       =>  menuju ke akhir huruf pada kata
c       =>  menuju ke change mode
superb  =>  ganti ke superb
\end{verbatim}


Contoh \emph{line visual mode}:

Asumsikan kita mempunyai baris teks sebagai berikut:

\begin{verbatim}
1 import os
2 from bottle import route, run
3
4 @route('/')
5 def homepage():
6   return 'Hello World!'
7
8 run(host='localhost', port=8080)
\end{verbatim}

Katakanlah Anda ingin menyorot baris 4 sampai 6, dengan Vim
Anda dapat melakukannya dengan

\begin{verbatim}
:4      => menuju ke baris ke 4
V       => sorot baris tersebut
2j      => sorot 2 baris dibawahnya
\end{verbatim}

Jika Anda perhatikan ada perubahan warna pada daerah yang
sedang Anda sorot, selanjutnya terserah Anda, apakah ingin
disalin (menggunakan tombol \verb=y=) atau dihapus (potong)
(menggunakan tombol \verb=d=).

Contoh \emph{block visual mode}:

Dengan \emph{visual block mode}, Anda dapat melakukan
berbagai langkah manipulasi teks dengan lebih cepat.
Perhatikan contoh berikut (dikombinasikan dengan perintah
\verb=I= (\emph{insert}):

\begin{verbatim}
Ini contoh1.txt
Ini contoh2.txt
Ini contoh3.txt
Ini contoh4.txt
Ini contoh5.txt
\end{verbatim}

Anda ingin menambahkan kata berkas pada setiap baris:

\begin{verbatim}
^       =>  menuju ke awal baris
w       =>  bergerak satu word
Ctrl+v  => mengaktifkan blok sorot
4j      =>  ke bawah 4 baris
I       =>  berubah ke mode insert
berkas  =>  ketik kata berkas
space   =>  memberi jarak 1 spasi dengan kata berikutnya
Esc     =>  kembali ke command mode
\end{verbatim}

Dan hasilnya:

\begin{verbatim}
Ini berkas contoh1.txt
Ini berkas contoh2.txt
Ini berkas contoh3.txt
Ini berkas contoh4.txt
Ini berkas contoh5.txt
\end{verbatim}

\emph{Amazing} ....

Contoh lain

Misal Anda memiliki baris kalimat seperti berikut:

\begin{verbatim}
Ini baris yang panjang
Pendek
Ini baris yang panjang
\end{verbatim}

Anda ingin menambahkan kata \verb=sangat= antara kata
\verb=yang= dengan kata \verb=panjang=.

\begin{verbatim}
^       =>  menuju ke awal baris
3w      =>  menuju ke kata ke-3
Ctrl+v  =>  aktifkan sorot blok
2j      =>  menuju 2 baris kebawah
I       =>  berubah ke insert mode
sangat  =>  ketik sangat
space   =>  memberi spasi antara kata
Esc     =>  kembali ke command mode
\end{verbatim}

Dan hasilnya:

\begin{verbatim}
Ini baris yang sangat panjang
Pendek
Ini baris yang sangat panjang
\end{verbatim}

Kita dapat lihat, dengan perintah \verb=I=, baris kedua
tidak berubah, karena memiliki panjang baris yang tidak
sama.

Contoh berikutnya dikombinasikan dengan perintah \verb=c=
(\emph{change}):

\begin{verbatim}
Ini baris yang sangat panjang
Pendek
Ini baris yang sangat panjang
\end{verbatim}

Mari kita ubah kata \verb=sangat= menjadi \verb=--SANGAT--=

\begin{verbatim}
^           =>  menuju ke awal baris
3w          =>  menuju ke kata ke-3
Ctrl+v      =>  aktifkan blok sorot
2j          =>  menuju ke 2 baris dibawahnya
e           =>  menuju ke akhir kata sangat
c           =>  change
--SANGAT--  =>  ketik kata pengganti
Esc         =>  kembali ke command mode
\end{verbatim}

Dan hasilnya:

\begin{verbatim}
Ini baris yang --SANGAT-- panjang
Pendek
Ini baris yang --SANGAT-- panjang
\end{verbatim}

Contoh berikutnya merupakan kombinasi antara \emph{visual
block} dengan perintah \verb=A= (\emph{append}):

\begin{verbatim}
Ini baris yang panjang
Pendek
Ini baris yang panjang
\end{verbatim}

Anda ingin menambahkan kata \verb=sangat= di antara kata
\verb=yang= dengan kata \verb=panjang=.

\begin{verbatim}
^           =>  menuju ke awal baris
3w          =>  menuju ke 3 kata ke kanan
h           =>  ke kiri 1 karakter
Ctrl+v      =>  aktifkan blok sorot
2j          =>  menuju 2 baris dibawahnya
A           =>  append mode
sangat      =>  ketik sangat
space       =>  beri jarak antar kata
Esc         =>  kembali ke command mode
\end{verbatim}

Dan hasilnya, kata \verb=sangat= ditambahkan pada setiap
baris! Ini adalah perbedaan perintah \verb=A= dengan
perintah lainnya pada mode blok sorot.

\begin{verbatim}
Ini baris yang sangat panjang
Pendek         sangat 
Ini baris yang sangat panjang
\end{verbatim}

Contoh berikut ini mengkombinasikan blok sorot dengan
perintah \verb=$= dan \verb=A= untuk menambahkan kata di
setiap baris yang ada.

\begin{verbatim}
Ini baris yang panjang
Pendek
Ini baris yang panjang
\end{verbatim}

Tambahkan kata \verb=sekali= di setiap baris

\begin{verbatim}
$       =>  menuju ke akhir baris
Ctrl+v  =>  aktifkan blok sorot
2j      =>  sorot 2 baris ke bawah
A       =>  append mode
Space   =>  beri jarak antar kata
sekali  =>  ketik sekali
Esc     =>  kembali ke command mode
\end{verbatim}

Dan hasilnya:

\begin{verbatim}
Ini baris yang panjang sekali
Pendek sekali
Ini baris yang panjang sekali
\end{verbatim}

Contoh berikutnya merupakan kombinasi antara blok sorot
dengan perintah \verb=r= (\emph{replace}).

\begin{verbatim}
Ini baris yang panjang
Pendek
Ini baris yang panjang
\end{verbatim}

Mari kita ubah kata \verb=panjang= menjadi kata
\verb=xxxxxx=

\begin{verbatim}
^           =>  menuju ke awal baris
3w          =>  ke kanan 3 kata
Ctrl+v      =>  aktifkan blok sorot
2j          =>  sorot 2 baris ke bawah
e           =>  ke akhir kata
r           =>  replace mode
x           =>  ganti ke huruf x
\end{verbatim}

Dan hasilnya:

\begin{verbatim}
Ini baris yang xxxxxxx
Pendek
Ini baris yang xxxxxxx
\end{verbatim}

Contoh berikutnya menggeser bagian dari baris agar sejajar
dengan baris di bawah (atas) nya.

\begin{verbatim}
Nama    :   Sopier
Alamat  :   Jogja
Pekerjaan   :   Wiraswasta
Pendidikan  :   Sarjana
\end{verbatim}

Anda ingin menggeser 2 baris paling atas, supaya tanda
\verb=:= sejajar dengan baris dibawahnya:

\begin{verbatim}
^       =>  menuju ke awal baris
f:      =>  menuju ke tanda :
Ctrl+v  =>  aktifkan blok sorot
j       =>  sorot 1 baris dibawahnya
>       =>  geser satu tab ke kanan
\end{verbatim}

Dan hasilnya:

\begin{verbatim}
Nama        :   Sopier
Alamat      :   Jogja
Pekerjaan   :   Wiraswasta
Pendidikan  :   Sarjana
\end{verbatim}

Anda juga dapat menggeser ke kiri dengan mengganti tanda
\verb=>= menjadi \verb=<=.

Tips:

Anda dapat melakukan sorot ulang teks yang sudah Anda sorot
sebelumnya dengan menggunakan perintah \verb=gv=.

\subsection{Fungsi Hapus / Potong}

Vim memiliki fungsi hapus yang sangat handal dan efisien,
perhatikan kombinasi tombol berikut:

\begin{verbatim}
x   =>   menghapus 1 karakter pada posisi kursor
X   =>   menghapus 1 karakter di depan posisi kursor
d   =>   menghapus daerah yang sedang disorot
dd  =>   menghapus satu baris
dw  =>   menghapus satu kata ke depan
db  =>   menghapus satu kata ke belakang
D   =>   menghapus dari posisi kursor sampai ke akhir baris
\end{verbatim}

\subsection{Undo dan Redo}
Berikut ini tombol perintah untuk melakukan \emph{undo} dan
\emph{redo} pada Vim:

\begin{verbatim}
u       =>   Undo
U       =>   Undo semua perubahan pada baris
Ctrl+R  =>   Redo
:e!     =>   Membatalkan semua perubahan pada berkas
\end{verbatim}

\subsection{Word Completion}
Sebagian orang mungkin menyebutnya \emph{auto completion},
apa pun itu, cara kerjanya adalah dengan mengetikkan
beberapa karakter kemudian Vim akan mencari dalam dokumen,
kata yang cocok dengan sekumpulan karakter tersebut.

Berikut contohnya:

\begin{verbatim}
Yang menjadikan momok bagi sebagian orang mungkin adalah
kenyataan bahwa di Vim kita dituntut untuk sesedikit 
mungkin menggunakan mouse.
\end{verbatim}

Semisal Anda ingin mengetik ulang kata \verb=mouse=,
daripada harus mengetik secara utuh Anda cukup mengetik
\verb=mo= diikuti dengan tombol \verb=Ctrl+p=, nanti Vim
akan menunjukkan kata apa saja yang diawali dengan
\verb=mo=.

\vspace{12pt}

\includegraphics[scale=0.6]{vim7.jpg}

Anda dapat juga menggunakan kombinasi tombol \verb=Ctrl+n=
untuk pencarian ke bawah (\emph{bottom}).

\begin{verbatim}
Ctrl+p      =>  pencarian ke atas (up)
Ctrl+n      =>  pencarian ke bawah (bottom)
\end{verbatim}

\subsection{Filename Completion}
Sebagai seorang \emph{developer}, Anda mungkin pernah
dihadapkan pada sebuah kode di mana Anda harus menulis
\emph{filepath} ke dalam kode tersebut. Vim memberikan
kemudahan untuk itu, tanpa kita harus mengingat-ingat di
mana berkas tersebut berada.

Perhatikan contoh kode berikut:

\begin{verbatim}
<html>
<head>
    <link rel="stylesheet" href="somefilepath.css"/>
</head>
<body>
</body>
</html>
\end{verbatim}

Dengan Vim, Anda dapat menyunting \verb=somefilepath.css=
dengan cara berikut:

Pastikan Anda berada pada baris yang akan disunting

\begin{verbatim}
^
f";;
ci"
[opsional] Ketik direktori di mana Anda mau mencari (misal /home/)
Ctrl+x Ctrl+f
Tekan Ctrl+f atau Ctrl+n untuk bergerak maju (forward)
Tekan Ctrl+p untuk bergerak mundur (backward)
Jika sudah, tekan Ctrl+x lagi
\end{verbatim}

Berikut ini tampilan ketika kita sedang menggunakan fitur
\emph{filename completion} dalam Vim.

\vspace{12pt}

\includegraphics[scale=0.6]{vim11.png}

\vspace{12pt}

\subsection{Replace Karakter}
Biasanya, ketika kita ingin mengganti sebuah karakter, kita
menghapus karakter lama, masuk ke \emph{insert mode},
kemudian memasukkan karakter yang baru. Kita dapat
mempersingkat langkah tersebut dengan menggunakan tombol
\verb=r= diikuti dengan karakter pengganti.

Contoh pada kata \verb=Vim=, kita ingin mengganti \verb=V=
kapital menjadi \verb=v= kecil, caranya tempatkan kursor
pada \verb=V=, tekan \verb=r=, kemudian \verb=v=, dan
karakter pun berubah.

Anda dapat juga menggunakan tombol \verb=R= untuk mengganti
beberapa huruf sekaligus, perhatikan contoh berikut:

\begin{verbatim}
Mr. Joni sedang belajar Vim
\end{verbatim}

Ganti \verb=xxxx= menjadi Joni dengan mengetikkan perintah
berikut:

\begin{verbatim}
^       =>   menuju ke awal baris
W       =>   menuju ke huruf x pertama
R       =>   beralih ke replace mode
Joni    =>   ketik Joni untuk mengganti xxxx
\end{verbatim}

\emph{Another little improvement on speed typing :)}

\subsection{Menyunting Kata dalam tanda ' " (}
Anda mungkin pernah berhadapan dengan teks seperti berikut
ini:

\begin{verbatim}
<title>Judul</title>
\end{verbatim}

Anda ingin mengganti kata \verb=Judul=?

\begin{verbatim}
dit     =>  menghapus kata di dalam html tags
i       =>  beralih ke insert mode
\end{verbatim}

Atau Anda memiliki kalimat berbentuk kurang lebih seperti
berikut?

\begin{verbatim}
<a href="http://www.infotiket.com">Info Tiket</a>
\end{verbatim}

Anda ingin menyunting kata yang berada dalam tanda \verb="=?
Tempatkan kursor pada baris tersebut dan lakukan langkah
berikut:

\begin{verbatim}
di"     =>  hapus kata di dalam tanda "
ci"     =>  hapus kata di dalam tanda " dan beralih ke insert mode
\end{verbatim}

Untuk penyuntingan kalimat / kata di dalam tanda lainnya,
cukup ganti tanda \verb="= dengan tanda yang Anda inginkan,
misal (\verb=di(=, \verb=di[=, \verb=di{=, dst).

Selain menggunakan \verb=dit=, ada juga perintah:

\begin{verbatim}
cit     =>  change inside tags
yit     =>  copy inside tags
vitp    =>  visual inside tags then put
\end{verbatim}

Silakan coba, Anda pasti menyukainya...

\section{Fungsi Pencarian}

Saya yakin fungsi ini banyak dipakai ketika kita sedang
menyunting berkas. Bagaimana menggunakan fitur pencarian ini
pada Vim?

\subsection{Pencarian Case Insensitive}

Vim pada dasarnya dapat melakukan pencarian pada satu
berkas, banyak berkas atau pada daerah tertentu yang kita
inginkan.

Perintah dasarnya adalah menggunakan tombol \verb=/= diikuti
dengan kata yang ingin kita cari, kemudian secara otomatis
Vim akan mencari dan meng-\emph{highlight} kata tersebut
(jika ada).

Contoh:

\begin{verbatim}
/Vim    =>  mencari kata Vim forward
Enter   =>  enter
n       =>  next match
N       =>  previous match
\end{verbatim}

Anda juga dapat menggunakan tombol \verb=?= untuk pencarian
kebelakang (\emph{backward}), berikut contohnya:

\begin{verbatim}
?Vim    =>  mencari kata Vim backward
Enter   =>  enter
n       =>  previous match
N       =>  next match
\end{verbatim}

Anda juga dapat melakukan pencarian menggunakan tombol
\verb=*=, caranya tempatkan kursor Anda pada kata yang ingin
Anda cari (misal \verb=Vim=), kemudian tekan tombol
\verb=*=, maka secara otomatis Vim akan melakukan pencarian
kata yang cocok dengan kata pada posisi kursor.

Tombol \verb=#= memiliki fungsi yang sama dengan \verb=*=,
hanya saja modus pencarian untuk \verb=#= adalah ke belakang
(\emph{backward}). Anda juga dapat menggunakan \verb=g*=
atau \verb=g#= untuk mencari kata yang \emph{non-exact}.

Sebagai contoh:

\begin{verbatim}
kata
katak
\end{verbatim}

Anda ingin mencari semua kata yang mengandung kata
\verb=kata= di dalamnya? Lakukan perintah berikut:

Letakkan kursor pada kata \verb=kata=, kemudian:

\begin{verbatim}
g*
\end{verbatim}

Maka semua kata yang mengandung kata \verb=kata= akan di
\emph{highlight} oleh Vim.

\subsection{Pencarian Case Sensitive}

Secara \emph{default}, Vim menggunakan modus pencarian
dengan pengaturan \emph{case insensitive}, ini artinya kata
\verb=Vim= dengan \verb=vim= dipandang sebagai satu kata
yang sama. Pertanyaan selanjutnya bagaimana cara kita
mencari dengan modus \emph{case sensitive} dengan Vim?

Terlebih dahulu, Anda dapat mengaktifkan modus
\verb=smartcase= pada Vim dengan mengetikkan perintah
berikut:\footnote{Saya juga menambahkan baris tersebut pada
berkas .vimrc pada direktori home}

\begin{verbatim}
:set smartcase
\end{verbatim}

Kemudian lakukan perintah berikut:

\begin{verbatim}
/vim\C
atau
/\Cvim
\end{verbatim}

Perintah di atas berarti mencari kata \verb=vim=, bukannya
\verb=Vim=.

\subsection{Fungsi Cari dan Ganti}
Apabila Anda ingin melakukan fungsi cari dan ganti pada
satu berkas utuh, Anda dapat menjalankan perintah berikut:

\begin{verbatim}
:%s/kata_asal/kata_ganti/g
\end{verbatim}

Atau jika Anda ingin membatasi pencarian hanya pada baris
tertentu:

\begin{verbatim}
:420, 421s/Anda/Kami/g
\end{verbatim}

Perhatikan tanda spasi di belakang koma.

Atau Anda hanya ingin cari dan ganti pada satu baris saja?

\begin{verbatim}
:s/Anda/Kami/g
\end{verbatim}

Atau mungkin Anda lebih suka sistem sorot daerah tertentu
kemudian baru melakukan cari dan ganti pada daerah yang Anda
sorot? 

\begin{verbatim}
V
5j
:s/Anda/Kami/g
\end{verbatim}

Perintah di atas berarti, sorot baris pada posisi kursor
sampai 5 baris di bawah posisi kursor, kemudian cari kata
\verb=Anda= dan ganti dengan kata \verb=Kami=.

Tanda \verb=%= berarti melakukan pencarian pada seluruh
baris di dokumen, jika Anda ingin melakukan fungsi cari dan
ganti pada satu baris saja, maka hilangkan tanda \verb=%=.

Tanda \verb=g= berarti melakukan fungsi ini pada semua
keterulangan (\emph{occurences}) pada baris. Jika tidak
menggunakan tanda \verb=g=, maka Vim hanya akan mengganti
kata pertama yang ditemukan pada baris.

Contoh:

\begin{verbatim}
aku dan kau bagaikan langit dan bumi
\end{verbatim}

Kemudian ketikkan perintah berikut

\begin{verbatim}
V               =>  sorot baris
:s/dan/dengan/  =>  ganti kata dan yang pertama saja
\end{verbatim}

Dan hasilnya:

\begin{verbatim}
aku dengan kau bagaikan langit dan bumi
\end{verbatim}

Bandingkan jika kita menggunakan tanda \verb=g=.

\begin{verbatim}
V               =>  sorot baris
:s/dan/dengan/g =>  ganti semua kata dan pada baris
\end{verbatim}

Dan hasilnya:

\begin{verbatim}
aku dengan kau bagaikan langit dengan bumi
\end{verbatim}

Anda juga dapat menentukan apakah fungsi ini bersifat
\emph{case sensitive} atau \emph{case insensitive}, secara
\emph{default}, Vim menggunakan sifat \emph{case
insensitive}, jika Anda ingin melakukan secara \emph{case
sensitive}, Anda dapat menambahkan penanda \verb=I=.
Perhatikan perintah berikut:

\begin{verbatim}
:%s/anda/kami/gI
\end{verbatim}

Perintah di atas hanya akan merubah kata \verb=anda=, tapi
tidak dengan kata \verb=Anda=.

Dengan menggunakan penanda \verb=c=, maka Anda akan
dihadapkan pada konfirmasi interaktif, apakah Anda akan
melakukan penggantian pada kata yang sudah ditemukan.

\begin{verbatim}
:%s/anda/kami/gcI
\end{verbatim}

Perintah di atas berarti, cari di seluruh dokumen, di
seluruh baris, kata \verb=anda=, dan ganti menjadi
\verb=kami=, dengan sebelumnya menanyakan konfirmasi pada
Anda, dengan format kurang lebih seperti ini:

\begin{verbatim}
replace with kami (y/n/a/q/l/^E/^Y)?
\end{verbatim}

Selanjutnya Anda dapat menekan \verb=y= untuk \emph{yes},
\verb=n= untuk \emph{no}, dan seterusnya...

Berikut ini bentuk perintah untuk mencari kata secara tepat
(\emph{exact match}) pada Vim. Mari kita gunakan contoh
berikut:

\begin{verbatim}
andai
seandainya
andaikan
\end{verbatim}

Jika Anda menggunakan perintah

\begin{verbatim}
:%s/andai/jika/g
\end{verbatim}

Maka hasilnya:

\begin{verbatim}
jika
sejikanya
jikakan
\end{verbatim}

Dan bisa dibilang, hasilnya kacau. Kita dapat menggunakan
pencarian dengan modus \emph{exact match} untuk mengatasi
hal ini.

Pencarian modus \emph{exact match} menggunakan bentuk
sebagai berikut

\begin{verbatim}
:%s/\<kata_yang_dicari\>/kata_ganti/g
\end{verbatim}

Perhatikan kita menambahkan tanda \verb=\<= dan \verb=\>=
pada awal dan akhir kata yang ingin kita cari. Sehingga
perintah di atas kita ubah menjadi sebagai berikut:

\begin{verbatim}
:%s/\<andai\>/jika/g
\end{verbatim}

Dan hasilnya:

\begin{verbatim}
jika
seandainya
andaikan
\end{verbatim}

Anda juga dapat melakukan fungsi cari dan ganti
dikombinasikan dengan pencarian pola dengan 
\verb=regular expression=, namun materi tersebut tidak akan
dibahas dalam tutorial ini, jika berminat silakan
di-\emph{explore} sendiri.

\section{Macro}

\subsection{Dasar Macro}
Fitur ini adalah fitur yang sangat-sangat saya sukai, karena
sesuai dengan prinsip DRY (\emph{Dont Repeat Yourself}.
Ambil contoh teks html berikut ini:

\begin{verbatim}
<ul>
    satu
    dua
    tiga
    empat
    lima
    enam
    tujuh
    delapan
    sembilan
    sepuluh
</ul>
\end{verbatim}

Semisal Anda ingin menambahkan \verb=tags <li>= dari satu
sampai sepuluh, daripada bercapek menambahkan satu-persatu,
Anda dapat membuat \verb=macro= kemudian menjalankan
\verb=macro= tersebut sebanyak yang Anda inginkan.

Tempatkan kursor pada kata \verb=satu=, kemudian jalankan
kombinasi perintah berikut:

\begin{verbatim}
qa
I
<li>
Esc
A
</li>
Esc
j
q
9@a
\end{verbatim}

Hasil:

\begin{verbatim}
<ul>
    <li>satu</li>
    <li>dua</li>
    <li>tiga</li>
    <li>empat</li>
    <li>lima</li>
    <li>enam</li>
    <li>tujuh</li>
    <li>delapan</li>
    <li>sembilan</li>
    <li>sepuluh</li>
</ul>
\end{verbatim}

\subsection{Melakukan Penomoran Secara Otomatis}

Ambil contoh berikut, saya ingin membuat 10 daftar kata
python dengan nomor berurutan dari 1 sampai 10. Dengan
bermodal satu baris berikut, saya dapat membuat 10 daftar
kata python lengkap dengan nomor yang berurutan.

\begin{verbatim}
1. python
\end{verbatim}

Berikut perintahnya di Vim:

\begin{verbatim}
qa
Y
p
Ctrl+A
q
8@a
\end{verbatim}

Dan hasilnya adalah sebagai berikut:

\begin{verbatim}
1. python
2. python
3. python
4. python
5. python
6. python
7. python
8. python
9. python
10. python
\end{verbatim}

\emph{So so so efficient, right?}

\section{Multi-tab}
Seringkali Anda harus bekerja dengan banyak berkas
sekaligus, di dunia \verb=IDE= Anda mungkin sudah tidak
asing lagi dengan fitur \emph{multi-tab}, di mana Anda dapat
membuka banyak berkas sekaligus dan berpindah antara satu
berkas dengan berkas lain semudah melakukan klik pada berkas
yang diinginkan.

Vim juga mengenal sistem \emph{tabbing} seperti itu, berikut
beberapa perintah ketika Anda bekerja dengan banyak
\emph{tab}:

\begin{verbatim}
:tabnew         =>  membuat tab baru
:tabnext        =>  berpindah ke tab berikutnya
:tabprev        =>  berpindah ke tab sebelumnya
:gt             =>  go to next tab
:gT             =>  go to prev tab
:tabfind        =>  mencari tab berdasar nama berkas
:tabclose       =>  menutup tab
\end{verbatim}

Ketika Anda mengaktifkan fitur \emph{tab}, maka pada layar
Vim bagian atas akan muncul \emph{tab} baru selayaknya
\emph{tab} yang Anda lihat pada \verb=IDE= lainnya, cuman
disini warnanya hitam dan putih :)

Untuk lebih lengkapnya, Anda dapat mengetikkan perintah
\verb=:tab= diikuti dengan tombol \verb=Tab= untuk melihat
perintah-perintah apa saja terkait dengan fitur ini.

\section{Registers}
Untuk meningkatkan efisiensi dalam pekerjaan penyuntingan
teks, Vim memiliki fitur \emph{registers}, di mana Anda
dapat menyimpan apa yang sudah Anda salin atau hapus ke
dalam sebuah \emph{key} tertentu.

Ketika sudah tersimpan, Anda dapat menambahkan apa yang
sudah Anda simpan atau menyalinnya ke tempat yang Anda
inginkan. 

Bentuk \emph{syntax} perintah \emph{registers} pada Vim
adalah sebagai berikut:

\begin{verbatim}
"kyy
\end{verbatim}

Perintah di atas berarti salin sebuah baris (\verb=y=)
kemudian simpan baris tersebut ke dalam tombol \verb=k=.
Jika Anda ingin menampilkan isi dari \emph{register}
tersebut, Anda dapat melakukan perintah berikut:

\begin{verbatim}
"kp
\end{verbatim}

Perintah tersebut berarti \emph{put} atau taruh isi dari
\emph{register} \verb=k= pada posisi kursor sekarang.

Anda dapat menambahkan isi sebuah \emph{register} dengan
menggunakan huruf kapital dari \emph{register} yang Anda
buat sebelumnya.

\begin{verbatim}
"Kyy
\end{verbatim}

Perintah di atas berarti salin baris pada posisi kursor,
kemudian tambahkan (\emph{append}) baris tersebut ke dalam
\emph{register} \verb=k=.

Perhatikan contoh berikut:

\begin{verbatim}
Vim adalah program penyunting teks yang handal.
\end{verbatim}

Tekan \verb="kyy= untuk menyalin baris di atas.

Kemudian Anda memiliki sebuah baris baru lagi

\begin{verbatim}
Namun, proses belajar Vim memang tidak mudah
\end{verbatim}

Berikutnya tekan \verb="Kyy= untuk menambahkan baris ini ke dalam
\emph{register} \verb=k=.

Kemudian \emph{put} isi dari \emph{register} \verb=k= ke
dalam sebuah baris, dan hasilnya:

\begin{verbatim}
Vim adalah program penyunting teks yang handal.
Namun, proses belajar Vim memang tidak mudah
\end{verbatim}

Selain menyimpan salinan, Anda dapat juga menyimpan hapusan
ke dalam \emph{register}, caranya tentu dengan mengganti
perintah salin dengan perintah hapus. Perhatikan contoh
perintah berikut:

\begin{verbatim}
"kdd    =>  hapus dan simpan sebuah baris ke dalam register k
"Kdd    =>  tambahkan hapusan berikutnya ke dalam register k
"kp     =>  taruh isi dari register k ke posisi kursor sekarang
\end{verbatim}

\section{Marks}
Sesuai dengan artinya, \emph{Marks} pada Vim berfungsi
sebagai penanda posisi, sehingga Anda dapat dengan mudah
menuju kembali ke posisi tersebut.

\emph{Marks} pada Vim disimbolkan dengan huruf alfabet dari
a--z untuk tiap berkas, dan huruf kapital A--Z untuk penanda
global. Jika Anda sedang menyunting 10 berkas, tiap berkas
dapat memiliki penanda posisi a, namun hanya memiliki 1 penanda
posisi A.

Perintah untuk mengaktifkan penanda pada posisi kursor
adalah dengan menekan tombol \verb=m= diikuti dengan huruf
sebagai penanda.

Misal:

\begin{verbatim}
ma  =>  beri tanda pada posisi kursor sekarang dengan huruf a sebagai penanda
\end{verbatim}

Untuk kembali pada posisi tersebut, tekan tanda petik
tunggal (\verb='=) atau tanda \emph{backtick} (\verb=`=).

\begin{verbatim}
'a  =>  menuju ke awal baris di mana penanda berada
`a  =>  menuju tepat ke posisi kursor di mana penanda berada
\end{verbatim}

Selanjutnya Anda pun dapat menyalin, menghapus atau pun
mengubah teks dengan penanda ini sebagai tujuan akhir.
Misalnya:

\begin{verbatim}
Vim merupakan aplikasi penyunting teks yang hebat.
\end{verbatim}

Jalankan perintah berikut:

\begin{verbatim}
^       =>  menuju ke awal baris
fp;;    =>  menuju ke huruf p pada kata penyunting
ma      =>  beri tanda a pada posisi ini
^       =>  kembali ke awal baris
d`a     =>  hapus dari posisi kursor sampai penanda a
\end{verbatim}

Beberapa perintah lain dari fungsi \emph{marks} pada Vim:

\begin{verbatim}
:marks          =>  daftar semua penanda yang aktif
:marks aB       =>  daftar penanda a, B
:delmarks a     =>  hapus penanda a
:delmarks a-d   =>  hapus penanda a,b,c,d
:delmarks!      =>  hapus semua penanda huruf kecil
\end{verbatim}

Fungsi navigasi dengan penanda:

\begin{verbatim}
]'          =>  menuju ke baris penanda berikutnya
['          =>  menuju ke baris penanda sebelumnya
]`          =>  menuju ke posisi kursor penanda berikutnya
[`          =>  menuju ke posisi kursor penanda sebelumnya 
\end{verbatim}

Yang perlu diingat, gunakan \verb='= (tanda petik tunggal)
untuk menuju ke awal baris di mana penanda berada, atau
gunakan \verb=`= (\emph{backticks}) untuk menuju ke posisi
kursor di mana penanda berada.\footnote{tanda backtick
berada di sebelah kiri angka 1 pada keyboard}

\section{Buffer}
Satu lagi fitur yang handal dari Vim untuk bekerja dengan
banyak berkas adalah \emph{buffer}. Saya sendiri lebih
menyukai ini dibandingkan dengan sistem \emph{tabbing},
karena layar kita tetap bersih, seolah-olah bekerja dengan
satu berkas, padahal sebenarnya banyak berkas yang sedang
kita sunting.

Berikut beberapa perintah terkait dengan \emph{buffer}:

\begin{verbatim}
:badd               =>   menambahkan berkas / buffer baru
:ls                 =>   melihat berkas-berkas yang sedang kita sunting
:bd                 =>   menghapus buffer (bukan menghapus berkas)
:b <angka>          =>   berpindah ke buffer <angka> sesuai dengan urutan pada perintah ls
:b <nama berkas>    =>   berpindah ke buffer berdasar nama
:bn                 =>   berpindah ke next buffer
:bp                 =>   berpindah ke prev buffer
\end{verbatim}

Berikut ini contoh perintah \verb=:ls=:

\begin{verbatim}
:ls
1 %a + "vim_docs.tex"                 line 593
2 #h   "~/.vimrc"                     line 1
3  h   "[No Name]"                    line 0
\end{verbatim}

Begini cara baca keluaran dari perintah \verb=:ls= di atas:

\begin{verbatim}
%           =>  buffer aktif yang sedang dilihat
#           =>  alternate buffer, tekan Ctrl+^ untuk berpindah ke alternate buffer
h           =>  hidden buffer (tidak sedang dilihat)
+           =>  ada perubahan dan belum disimpan
-           =>  inactive buffer
line xxx    =>  menunjukkan di baris berapa kursor Anda berada
\end{verbatim}

Terlihat di sana saya sedang menyunting 3 berkas, di mana
berkas yang aktif saya sunting saat ini adalah berkas nomor
1 (ditandai dengan \verb=%=).

Kalau saya ingin pindah ke berkas \verb=.vimrc=, saya
tinggal perintahkan \verb=:b 2= atau \verb=:b vimrc= atau
\verb=:bn=.

Selain menghapus satu per satu, Anda juga dapat menghapus
banyak \emph{buffer} sekaligus dengan menggunakan
\emph{range}, contoh:

\begin{verbatim}
:1,5bd      =>  menghapus buffer 1 sampai 5
\end{verbatim}

Jika Anda ingin menghapus sebuah \emph{buffer} tanpa
melakukan penyimpanan, Anda dapat menambahkan tanda perintah
\verb=!= seperti pada contoh berikut:

\begin{verbatim}
:bd!
\end{verbatim}

Anda dapat juga melakukan pencarian ke dalam banyak
\emph{buffer} sekaligus, atau mungkin cari dan ganti ke
banyak \emph{buffer} sekaligus. Selanjutnya, silakan
di-\emph{explore} sendiri kemampuan dari \emph{buffer} ini.

\section{Code Folding}
Bagi Anda yang menulis ribuan baris teks atau pun kode,
fitur ini pasti sangat berguna buat Anda untuk membuat
tulisan atau kode Anda ``lebih enak`` dilihat.

Berikut ini tampilan kode sebelum dan sesudah menggunakan
\emph{folding} di Vim:

\vspace{12pt}

\includegraphics[scale=0.6]{vim12.png}

\vspace{12pt}

\includegraphics[scale=0.6]{vim13.png}

\vspace{12pt}

Vim memiliki 6 cara dalam melakukan \emph{folding}:

\begin{verbatim}
1. manual
2. indent
3. syntax
4. marker
5. expr
6. diff
\end{verbatim}

Untuk sementara ini mari kita pelajari 3 dari 6 cara di
atas, yakni cara \verb=manual=, \verb=marker= dan
\verb=indent=.

\emph{Folding} \verb=manual= dapat Anda lakukan dengan
terlebih dahulu menyorot baris yang ingin Anda lipat,
kemudian lipat (buka lipatan) dengan menekan:

\begin{verbatim}
zf  =>  melipat baris
za  =>  membuka lipatan
\end{verbatim}

Contoh:

\begin{verbatim}
Jadual hari ini:
1. Rapiin kamar kerja
2. Cuci motor
3. Mandi
4. Sarapan
5. Coding
6. Makan siang
7. Ngopi2 sama temen
\end{verbatim}

Posisikan kursor pada jadual pertama, kemudian tekan:

\begin{verbatim}
V   =>  sorot baris pertama
6j  =>  sorot 6 baris dibawahnya
zf  =>  lipat baris
\end{verbatim}

Atau cara lebih singkat:

\begin{verbatim}
zf6j
\end{verbatim}

Selain dengan teknik sorot, Anda pun dapat menggunakan
teknik \verb=marks= yang dapat Anda pelajari di bagian lain
tutorial ini.

Menggunakan contoh di atas, kita dapat memanfaatkan fitur
\verb=marks= untuk melakukan pelipatan:

Posisikan kursor pada jadual nomer 7, kemudian tekan:

\begin{verbatim}
ma      =>  beri tanda dan simpan ke register a
6k      =>  menuju ke jadual nomer 1
zf'a    =>  lipat baris sampai mark a
\end{verbatim}

Berikutnya pelipatan dengan metode \verb=marker=, sebelumnya
ketik perintah berikut ini:

\begin{verbatim}
:set foldmethod=marker
\end{verbatim}

Selanjutnya Anda dapat memberikan tanda khusus pada
bagian-bagian yang ingin Anda lipat. Perhatikan kode
\verb=LaTeX= berikut:

\begin{verbatim}
\section{Pendahuluan}
%{{{
Tulisan-tulisan ini sebenarnya bukanlah tutorial lengkap
yang mengajarkan kepada Anda seluk-beluk program Vim,
melainkan sekedar catatan pribadi penulis yang coba
dituangkan kedalam sebuah berkas elektronik dengan tujuan
untuk dokumentasi pribadi, syukur-syukur kalau ada pihak
lain yang membacanya dan mampu mendapatkan manfaat dari
tulisan-tulisan ini.
%}}}
\end{verbatim}

Kemudian tambahkan penanda (sebagai contoh 
\verb=%{{{ %}}}=):

\begin{verbatim}
\section{Pendahuluan}
%{{{
Tulisan-tulisan ini sebenarnya bukanlah tutorial lengkap
yang mengajarkan kepada Anda seluk-beluk program Vim,
melainkan sekedar catatan pribadi penulis yang coba
dituangkan kedalam sebuah berkas elektronik dengan tujuan
untuk dokumentasi pribadi, syukur-syukur kalau ada pihak
lain yang membacanya dan mampu mendapatkan manfaat dari
tulisan-tulisan ini.
%}}}
\end{verbatim}

Secara otomatis, Vim akan melihat tanda tersebut, letakkan
kursor di dalam penanda tersebut, kemudian tekan:

\begin{verbatim}
za  =>  melipat kode
za  =>  lakukan lagi untuk membuka lipatan
\end{verbatim}

Atau Anda pun dapat melipat semua bagian yang sudah Anda
beri penanda dengan mengetikkan:

\begin{verbatim}
zM      =>  melipat semua bagian yang sudah diberi marker
\end{verbatim}

dan untuk membukanya:

\begin{verbatim}
zR      =>  membuka semua lipatan yang sudah diberi marker
\end{verbatim}

Anda dapat menentukan sendiri \verb=marker= yang akan
dipakai, dengan memberikan perintah:

\begin{verbatim}
:set foldmarker=/*,*/
\end{verbatim}

Perintah di atas, berarti \verb=marker= pembuka ditandai
dengan tanda \verb=/*= dan \verb=marker= penutup dengan
tanda \verb=*/=.

Metode berikutnya adalah dengan metode \verb=indent=. Teknik
ini berguna ketika Anda memiliki kode yang memiliki
indentasi terstruktur (misal bahasa \verb=Python=).

Sebelumnya ketik perintah berikut:

\begin{verbatim}
:set foldmethod=indent
\end{verbatim}

Perhatikan kode \verb=Python= berikut:

\begin{verbatim}
class C(object):

    nama = 'sopier'
    hobi = 'plesir'

    def __init__(self):
        pass
\end{verbatim}

Anda dapat melipat kode, dengan menempatkan kursor pada kode
tersebut, kemudian tekan

\begin{verbatim}
za  =>  melipat kode
za  =>  lakukan lagi untuk membuka lipatan
\end{verbatim}

Atau Anda dapat melipat semua baris yang memiliki indentasi
dengan menekan:

\begin{verbatim}
zM  =>  melipat semua baris yang memiliki indent
zR  =>  membuka semua lipatan
\end{verbatim}

Anda dapat mempelajari lebih dalam lagi dengan menekan
perintah 

\begin{verbatim}
:help folding
\end{verbatim}

\section{File Explorer}
Salah satu fungsi penting sebuah program penyunting teks
adalah kemampuan jelajah berkas, jika Anda terbiasa dengan
program berbasis \emph{GUI}, Anda tinggal cari menu
\emph{Open} dan mulai mencari berkas yang Anda inginkan.
Bagaimana dengan Vim?

\vspace{12pt}

\includegraphics[scale=0.6]{vim8.png}

\vspace{12pt}

Secara \emph{default}, Vim memiliki kemampuan tersebut,
berikut ini kombinasi tombol perintah yang Anda perlukan
untuk mengakses kemampuan jejalah berkas pada Vim:

\begin{verbatim}
:edit .     =>  membuka file explorer
:e.         =>  membuka file explorer
:sp.        =>  membuka file explorer dengan horizontal split
:vs.        =>  membuka file explorer dengan vertical split
\end{verbatim}

\vspace{12pt}

\includegraphics[scale=0.6]{vim10.png}

\vspace{12pt}

Ketika Anda berada pada fitur jelajah berkas, Anda dapat
menggunakan kombinasi tombol berikut untuk membuat berkas
baru, direktori baru, mengganti nama, dan menghapus berkas:

\begin{verbatim}
%   =>  membuat berkas baru
d   =>  membuat direktori baru
R   =>  mengubah nama berkas / direktori pada kursor
D   =>  menghapus berkas / direktori pada kursor
\end{verbatim}

Karena fitur \emph{file explorer} tidak lain adalah
\emph{buffer}, maka Anda dapat dengan mudah menutupnya
dengan perintah \verb=:bd= atau melakukan pencarian
menggunakan \verb=/= (\emph{forward}), atau \verb=?=
(\emph{backward}). Dan tentunya juga tombol \verb=hjkl=
untuk melakukan navigasi ke kiri, bawah, atas dan kanan.

\vspace{12pt}

\includegraphics[scale=0.6]{vim9.png}

\vspace{12pt}

\section{Split Screen}
Dalam bekerja dengan banyak berkas, Vim juga memiliki
kemampuan untuk memecah layar menjadi beberapa bagian, baik
itu horisontal maupun vertikal.

Perhatikan gambar berikut:

\vspace{12pt}

\includegraphics[scale=0.6]{vim6.jpg}

\vspace{12pt}

Terlihat saya sedang menyunting 3 berkas dan ketiga-tiganya
terbuka, fitur ini sangat membantu ketika kita bekerja
dengan banyak berkas sekaligus. Bagaimana caranya:

\begin{verbatim}
:sp         =>  split horisontal
:vsp        =>  split vertical
Ctrl+w+w    =>  berpindah antar window
Ctrl+w+r    =>  berpindah antar window clockwise
Ctrl+w+R    =>  berpindah antar window counter-clockwise
Ctrl+w+l    =>  berpindah ke window sebelah kanan
Ctrl+w+h    =>  berpindah ke window sebelah kiri 
Ctrl+w+j    =>  berpindah ke window sebelah bawah 
Ctrl+w+k    =>  berpindah ke window sebelah atas 
Ctrl+w+-    =>  memperkecil ukuran window (mode horizontal)
Ctrl+w++    =>  memperbesar ukuran window (mode horizontal)
:q          =>  menutup window
\end{verbatim}

Anda dapat juga menggunakan perintah \verb=:sp= atau
\verb=:vsp= diikuti dengan nama berkas yang ingin Anda
sunting dalam \emph{window} baru.

\section{Session}
Fungsi fitur ini adalah untuk menyimpan berkas-berkas yang
Anda kerjakan sebelumnya, daripada membuka ulang
satu-persatu berkas tersebut, Anda tinggal menyimpannya ke
dalam \emph{session} untuk kemudian dibuka kembali, dan
otomatis Vim akan membuka berkas-berkas tersebut dalam
\emph{buffer}-nya. 

Berikut cara kita mengelola \emph{session} dalam Vim:

\begin{verbatim}
Menyimpan session
:mksession work1.vim

Memanggil session dari dalam Vim
:source work1.vim

Memanggil session dari terminal
$ vim -S work1.vim
\end{verbatim}


\section{Konfigurasi Vim}
Anda dapat mengatur program Vim dengan membuat berkas
\verb=.vimrc= pada direktori \verb=home= Anda.

\begin{verbatim}
$ vim ~/.vimrc
\end{verbatim}

Pengaturan ini sendiri mungkin berbeda antara pengguna satu
dengan pengguna lainnya, tergantung selera dan kebiasaan,
berikut konfigurasi yang saya pakai:

\begin{verbatim}
set nocompatible

filetype on
filetype plugin on
filetype indent on
syntax on

set autowrite
set ruler
set hidden
set autochdir

colorscheme delek 

set tabstop=8
set shiftwidth=4
set softtabstop=4
set expandtab

set showcmd
set number
set smartindent
set autoindent
set laststatus=2
set linespace=3

set wrap
set linebreak
set nolist
set incsearch
set hlsearch
set ignorecase
set smartcase

set foldenable
set mousehide
"set splitbelow

nmap <space> :

set wildmode=list:longest

imap jj <esc>

map <f2> :w\|!python %
\end{verbatim}

Konfigurasi ini secara garis besar adalah standar, kecuali
saya merubah tombol \verb=:= menjadi \verb=Space=, dan
tombol \verb=Esc= menjadi \verb=jj=, semua ini dilakukan
biar posisi tangan tidak bergeser ke kanan dan ke kiri.

Silakan dicoba, atau Anda mungkin memiliki preferensi
sendiri, Vim memberikan kebebasan untuk itu..

\section{Lain-lain}
Berikut ini kumpulan tips dan trik yang sering saya pakai
dan siapa tahu bermanfaat juga buat Anda...

\subsection{Menjalankan Perintah Shell dari Vim}
Ini termasuk salah satu fitur yang sangat saya sukai, Anda
tidak perlu bolak-balik keluar dari Vim untuk sekedar
menjalankan perintah \verb=Shell=.

Sebagai contoh, dokumen ini ditulis menggunakan
\verb=LaTeX=, dan tentunya saya sering melakukan
\verb=compiling= dari format \verb=.tex= ke \verb=.pdf=
untuk melihat apakah ada kesalahan penulisan atau tidak.
Proses \verb=compiling= itu sendiri menggunakan \verb=shell=
\emph{command}:

\begin{verbatim}
$ pdflatex vimdocs.tex
\end{verbatim}

Daripada harus keluar masuk Vim, saya dapat mengeksekusi
perintah tersebut dari dalam Vim menggunakan perintah
berikut:

\begin{verbatim}
:!pdflatex vimdocs.tex
\end{verbatim}

Dan jika ingin mengulangi perintah terakhir, asya dapat
dengan mudah mengetik berikut di Vim:

\begin{verbatim}
:!!
\end{verbatim}

Sangat efisien bukan?

\subsection{Menyisipkan Keluaran dari Shell Command ke dalam
Vim}

Semisal Anda ingin menyisipkan tanggal dan jam saat ini ke
dalam dokumen Vim yang sedang Anda tulis, daripada harus
capek-capek mengetik ulang, kita dapat menggunakan fungsi
\verb=read= berikut:

\begin{verbatim}
:r !date
\end{verbatim}

Atau ingin menyisipkan kalender menggunakan perintah
\verb=shell cal=?

\begin{verbatim}
:r !cal
\end{verbatim}

Maka, secara otomatis keluaran dari perintah \verb=date=
akan disisipkan pada baris di mana kursor Anda berada.

\begin{verbatim}
   December 2012      
Su Mo Tu We Th Fr Sa  
                   1  
 2  3  4  5  6  7  8  
 9 10 11 12 13 14 15  
16 17 18 19 20 21 22  
23 24 25 26 27 28 29  
30 31                 
\end{verbatim}

\subsection{Singkatan}
Anda memiliki kata yang sangat sering Anda ulang-ulang dan
ingin membuat versi singkatannya supaya lebih efisien dalam
proses pengetikan? Vim memiliki fitur tersebut dengan
perintah berikut:

\begin{verbatim}
:ab yg yang
\end{verbatim}

Maka, secara otomatis setiap kali Anda mengetik \verb=yg=
diikuti dengan \verb=spasi= atau \verb=,=, maka kata
tersebut berubah menjadi \verb=yang=.

Secara lengkap, berikut perintah-perintah terkait dengan
fitur \emph{abbreviations} ini.

\begin{verbatim}
:ab dg dengan   =>  dg diganti dengan
:ab             =>  daftar semua singkatan
:una dg         =>  hapus dg dari daftar singkatan
:abc            =>  hapus semua daftar singkatan
\end{verbatim}

\subsection{Membatasi Panjang Baris Maksimum n Karakter}
Baris yang terlalu panjang kadang merepotkan, karena kita
harus melakukan \emph{scrolling} ke kanan untuk dapat
membaca baris bagian kanan ujung. Untuk mengatasi hal
tersebut, saya sendiri lebih suka mengatur Vim dengan
panjang maksimum adalah 80 karakter:

\begin{verbatim}
:set textwidth=80
\end{verbatim}

Konfigurasi ini otomatis memerintahkan Vim untuk memotong
baris setiap kali kursor berada pada posisi ke-80.

Biar tidak berulang-ulang mengetikkan perintah tersebut,
kita dapat memasukkan konfigurasi tersebut kedalam berkas
\verb=.vimrc=:

\begin{verbatim}
set textwidth=80
\end{verbatim}

Selain cara otomatis tersebut, kita juga bisa memotong
secara \emph{manual} dengan mengetikkan perintah:

\begin{verbatim}
gql
\end{verbatim}

\subsection{Menggabung Dua atau Lebih Baris}
Lawan dari memotong baris yang panjang, adakalanya kita
ingin menggabungkan beberapa baris menjadi 1 baris saja.
Hal ini dapat kita lakukan dengan menempatkan kursor di
baris paling atas dari beberapa baris yang ingin Anda
gabung, kemudian ketikkan:

\begin{verbatim}
J       =>  jika hanya dua baris
JJ      =>  3 baris
JJJ     =>  4 baris
\end{verbatim}

Ulangi menekan \verb=J= sampai semua baris tergabung menjadi
satu. Perintah ini secara otomatis akan menambahkan spasi
diantara baris yang digabungkan. Jika Anda ingin menggabung
baris tanpa spasi, gunakan perintah \verb=gJ=.

Contoh:

\begin{verbatim}
Ini baris pertama
Ini baris kedua
Ini baris ketiga
Ini baris keempat
\end{verbatim}

Tekan perintah berikut:

\begin{verbatim}
^       =>  menuju ke awal baris
J       =>  gabung garis 1 dan 2
J       =>  gabung lagi
gJ      =>  gabung lagi, kali ini tanpa spasi
\end{verbatim}

Sehingga hasilnya sebagai berikut:

\begin{verbatim}
Ini baris pertama Ini baris kedua Ini baris ketigaIni baris keempat
\end{verbatim}

\subsection{Lowercase, Uppercase dan Titlecase}
Bagaimana cara mengubah dari \emph{lowercase} menjadi
\emph{uppercase} dalam Vim?

Perhatikan contoh kalimat berikut:

\begin{verbatim}
vim merupakan program penyunting teks yang hebat.
\end{verbatim}

Anda dapat merubah kalimat di atas menjadi \emph{uppercase}
dengan menekan tombol berikut:

\begin{verbatim}
V       =>   sorot baris
U       =>   ubah menjadi uppercase, atau
u       =>   ubah menjadi lowercase
\end{verbatim}

Berikut hasil ketika sudah dirubah menjadi \emph{uppercase}

\begin{verbatim}
VIM MERUPAKAN PROGRAM PENYUNTING TEKS YANG HEBAT.
\end{verbatim}

Untuk mengubah kata atau baris menjadi \emph{titlecase},
kita dapat menggunakan fitur \emph{search n replace}
digabungkan dengan \emph{regular expression}:

\begin{verbatim}
V               =>  sorot baris
:s/\w*/\u&/g    =>  ubah setiap awal kata menjadi huruf kapital
\end{verbatim}

Dan hasilnya:

\begin{verbatim}
Vim Merupakan Program Penyunting Teks Yang Hebat.
\end{verbatim}

Anda juga dapat melakukannya dengan cara \emph{semi-manual},
berikut caranya:

\begin{verbatim}
^   =>   menuju ke awal baris
vU  =>   sorot huruf pada awal kata dan ubah menjadi uppercase
w.  =>   menuju ke awal kata berikutnya dan ulangi langkah sebelumnya
w.  =>   menuju ke awal kata berikutnya dan ulangi langkah sebelumnya
ulangi lagi sebanyak yang Anda inginkan
\end{verbatim}

Atau Anda juga dapat memanfaatkan \emph{plugin} bernama
\verb=titlecase= yang dapat Anda unduh di http://www.vim.org

Selamat mencoba ...

\subsection{Cari dan Hapus Baris Berdasar Pola}
Saatnya kita mencoba mempelajari fitur
\emph{regular expression} pada Vim. Pada bahasan kali ini
kita akan mencari dan menghapus baris jika dalam tersebut
terdapat pola (\emph{pattern}) yang cocok dengan yang kita
tentukan.

Perhatikan contoh berikut ini:

\begin{verbatim}
this is a line
these are also lines
those are line
that is another line
\end{verbatim}

Kita dapat menghapus baris yang memiliki kata \verb=lines=
dengan mengetikkan perintah berikut:

\begin{verbatim}
:g/lines/d
\end{verbatim}

atau menghapus kata yang mengandung \verb=is=

\begin{verbatim}
:g/is/d
\end{verbatim}

atau kata yang diawali dengan \verb=th=

\begin{verbatim}
:g/^th/d
\end{verbatim}

atau kata yang diakhiri dengan \verb=ine=

\begin{verbatim}
:g/ine$/d
\end{verbatim}

Anda dapat juga menentukan pada baris berapa sampai berapa
fungsi ini diterapkan, contoh berikut akan menghapus baris
yang memiliki kata \verb=lines= pada baris 1296 sampai 1299:

\begin{verbatim}
:1296,1299g/lines/d
\end{verbatim}

\subsection{Membuat Baris Baru Identik}
Berikut ini adalah perintah untuk membuat banyak baris yang
identik:

\begin{verbatim}
5aVim is great!     =>  buat 5 baris baru dengan isi Vim is great!
Tekan enter         =>  buat baris baru
Tekan Esc           =>  kembali ke command mode
\end{verbatim}

Dan hasilnya:

\begin{verbatim}
Vim is great!
Vim is great!
Vim is great!
Vim is great!
Vim is great!
\end{verbatim}

\subsection{Cara Lain Beralih ke Command Mode}
Di bagian atas sudah disebutkan bahwa untuk beralih dari
\emph{insert mode} ke \emph{command mode}, kita dapat
menekan tombol \verb=Esc= pada \emph{keyboard}.
Kenyataannya, bagi sebagian orang (termasuk saya), tombol
\verb=Esc= terlalu jauh letaknya dari \emph{home row}.

Untuk itu sebagian pengguna Vim ada yang melakukan
\emph{remapping} tombol \verb=Esc= ke tombol lain, misalnya
saya \emph{remap} tombol \verb=Esc= ke tombol \verb=jj=.
Caranya adalah dengan menambahkan baris berikut pada berkas
\verb=.vimrc= Anda:

\begin{verbatim}
imap jj <esc>
\end{verbatim}

Selain itu, sebenarnya Anda dapat juga menggunakan tombol
\verb=Ctrl+[= atau \verb=Ctrl+c= untuk berpindah dari
    \emph{insert mode} ke \emph{command mode}.

Silakan pilih yang Anda suka, selama itu untuk meningkatkan
produktivitas Anda dalam pekerjaan penyuntingan teks.

\subsection{Bracket Matching}
Tanda kurung biasanya selalu ditulis dalam bentuk
berpasangan, ada \verb=(= maka ada \verb=)=, begitu juga
dengan tanda \verb={}= dan \verb=[]=.

Kalau Anda menulis kode program, biasanya pasangan dari
tanda tersebut berada jauh di bawah atau kanan, Vim dalam
hal ini memberikan kemudahan untuk mencari pasangan dari
tanda-tanda tersebut. Dengan menekan tanda \verb=%=, maka
secara otomatis kursor akan bergerak ke pasangan tanda
tersebut.

Lihat kode berikut:

\begin{verbatim}
require('casper').create({
    loadImages: false,
    verbose: true,
    logLevel: debug
});
\end{verbatim}

Posisikan kursor Anda pada kurung buka "\verb=(=" setelah kata
\verb=create=, kemudian tekan \verb=%=, apa yang terjadi?
Vim akan mencocokkan di mana letak kurung tutup yang
merupakan pasangan dari kurung buka tadi, tekan \verb=%=
sekali lagi, dan Anda berpindah ke tempat kursor Anda
sebelumnya.

Apa jadinya jika tidak ada pasangan kurung yang tepat? Vim
tidak akan memberitahukan pesan kesalahan apa-apa, namun
karena posisi kursor tidak berpindah, maka dapat disimpulkan
bahwa kurung tersebut belum ditutup.

\begin{verbatim}
require('casper').create({
    loadImages: false,
    verbose: true,
    logLevel: debug
};
\end{verbatim}

\subsection{Mengaktifkan Penomoran Baris}
Secara \emph{default}, Vim tidak menampilkan penomoran
baris, namun Anda dapat dengan mudah menampilkannya dengan
mengetikkan:

\begin{verbatim}
:set number     =>  nomor baris aktif
:set nonumber   =>  nomor baris tidak aktif 
\end{verbatim}

\subsection{Mengulang Perintah Terakhir}

Tahukah Anda kalau Anda dapat mengulang perintah Anda yang
terakhir kali menggunakan tanda titik (\emph{dot})?

Misal Anda memiliki teks sebagai berikut:

\begin{verbatim}
tahukah anda? vim sudah berumur 20 tahun lebih?
\end{verbatim}

Anda ingin mengganti kalimat tersebut menjadi
\emph{titlecase}, bagaimana caranya?

\begin{verbatim}
^   =>  menuju ke awal baris yang berisi huruf
v   =>  sorot huruf pertama pada kata pertama
U   =>  ubah menjadi huruf kapital
w   =>  bergerak ke kata berikutnya
.   =>  ulangi perintah vU
\end{verbatim}

Dan hasilnya...

\begin{verbatim}
Tahukah Anda? Vim Sudah Berumur 20 Tahun Lebih?
\end{verbatim}

Anda dapat juga mengulang perintah terakhir yang Anda
berikan ke Vim dengan mengetikkan:

\begin{verbatim}
@:
\end{verbatim}

Sebagai contoh, Anda ingin membuka \emph{tab} baru pada Vim
dengan menggunakan perintah \verb=:tabnew=. Sebentar
kemudian Anda butuh untuk membuka \emph{tab} baru satu lagi,
daripada harus mengetik \verb=:tabnew= lagi, Anda dapat
mengulang perintah tersebut dengan cara \verb=@:=, dan
perintah yang paling terakhir pun, dijalankan ulang oleh
Vim.

\begin{verbatim}
:tabnew     =>  membuka tab baru
@:          =>  buka tab baru lagi
:tabclose   =>  tutup tab 
@:          =>  ulangi perintah tutup tab
\end{verbatim}

\subsection{Eksekusi Kode Bash dalam Vim}

Selain dapat menjalankan kode Python, Vim juga dapat
menjalankan perintah Bash, sebagai contoh Anda memiliki
baris kode seperti berikut ini:

\begin{verbatim}
echo "hai from Bash"
date
echo "bye..."
\end{verbatim}

Kemudian jalankan perintah berikut:

\begin{verbatim}
V2j     =>  sorot baris pada kursor dan 2 baris dibawahnya
:!bash  =>  eksekusi kode
\end{verbatim}

dan hasilnya:

\begin{verbatim}
hai from Bash
Fri Dec 21 06:30:10 WIT 2012
bye...
\end{verbatim}

\subsection{Mengetahui Nama Berkas yang sedang disunting}
Daripada harus mengetik ulang nama berkas, Vim memberikan
kemudahan dengan menggunakan \verb=register %=, yang mana
fungsinya adalah sebagai \emph{shortcut} untuk berkas yang
sedang disunting.

Jika Anda ingin menampilkan nama berkas yang sedang
disunting, pada \emph{insert mode}, ketik:

\begin{verbatim}
Ctrl+r
%
\end{verbatim}

Atau, pada \emph{command mode}:

\begin{verbatim}
"%p
\end{verbatim}

dan hasilnya:

\begin{verbatim}
vim_docs.tex
\end{verbatim}

Dan jika Anda ingin menampilkan nama berkas secara utuh
dengan \emph{path}-nya, pada \emph{command mode}, ketik
perintah berikut:

\begin{verbatim}
:put =expand('%:p')
\end{verbatim}

Dan hasilnya:

\begin{verbatim}
/home/banteng/Dropbox/dokumentasi/vim/vim_docs.tex
\end{verbatim}

\subsection{Resize Splits}
Sebagaimana layaknya sebuah \emph{window} pada aplikasi
\emph{GUI} yang dapat diubah ukurannya dengan melakukan
\emph{click-and-drag}, Vim juga memiliki fitur tersebut
ketika kita bekerja dengan \emph{split screen}.

Perintah untuk mengubah ukuran layar horisontal berbeda
dengan perintah untuk layar vertikal. Berikut ini perintah
untuk mengubah ukuran layar pada mode \emph{split} di Vim:

Mode \emph{horizontal split}
\begin{verbatim}
Ctrl+w +    =>  Increase
Ctrl+w -    =>  Decrease
Ctrl+w _    =>  Maximize
Ctrl+w =    =>  Equal size
10Ctrl+w +  =>  Increase 10 lines
10Ctrl+w -  =>  Decrease 10 lines
\end{verbatim}

Mode \emph{vertical split}
\begin{verbatim}
Ctrl+w >    =>  Increase
Ctrl+w <    =>  Decrease
Ctrl+w |    =>  Maximize
Ctrl+w =    =>  Equal size
10Ctrl+w >  =>  Increase 10 chars
10Ctrl+w <  =>  Decrease 10 chars
\end{verbatim}

\end{document}
